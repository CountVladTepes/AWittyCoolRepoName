\documentclass{article}

\begin{document}
	\title{A \LaTeX document.}
	\date{}
	\maketitle
	
	This text is here for illustration of how using a new line for each sentence in your \texttt{.tex} document can be useful when using Git. As you can see in this paragraph we are typing each sentence one after the other. Although this is a natural way to type it is not good to identify exactly what has changed from one commit to the next. If I make a change (such as this one) to this document GitHub will highlight the whole paragraph.
	
	However, we can type a new sentence on a new line in \LaTeX and it will format it without a line break.
	This means we can type a new sentence on a new line and it will appear as if it has been typed in the usual way of one sentence followed by another.
	A new line.
	This is beneficial when using version control software as if you add/remove/change something in a sentence it will highlight only the line that has changed making it easy to identify changes (such as this one).
	If I make a change in this paragraph it will be easy to identify.
	
\end{document}